\documentclass[]{book}
\usepackage{lmodern}
\usepackage{amssymb,amsmath}
\usepackage{ifxetex,ifluatex}
\usepackage{fixltx2e} % provides \textsubscript
\ifnum 0\ifxetex 1\fi\ifluatex 1\fi=0 % if pdftex
  \usepackage[T1]{fontenc}
  \usepackage[utf8]{inputenc}
\else % if luatex or xelatex
  \ifxetex
    \usepackage{mathspec}
  \else
    \usepackage{fontspec}
  \fi
  \defaultfontfeatures{Ligatures=TeX,Scale=MatchLowercase}
\fi
% use upquote if available, for straight quotes in verbatim environments
\IfFileExists{upquote.sty}{\usepackage{upquote}}{}
% use microtype if available
\IfFileExists{microtype.sty}{%
\usepackage{microtype}
\UseMicrotypeSet[protrusion]{basicmath} % disable protrusion for tt fonts
}{}
\usepackage[margin=1in]{geometry}
\usepackage{hyperref}
\hypersetup{unicode=true,
            pdftitle={A tao of programming},
            pdfauthor={Dan MacLean},
            pdfborder={0 0 0},
            breaklinks=true}
\urlstyle{same}  % don't use monospace font for urls
\usepackage{natbib}
\bibliographystyle{apalike}
\usepackage{longtable,booktabs}
\usepackage{graphicx,grffile}
\makeatletter
\def\maxwidth{\ifdim\Gin@nat@width>\linewidth\linewidth\else\Gin@nat@width\fi}
\def\maxheight{\ifdim\Gin@nat@height>\textheight\textheight\else\Gin@nat@height\fi}
\makeatother
% Scale images if necessary, so that they will not overflow the page
% margins by default, and it is still possible to overwrite the defaults
% using explicit options in \includegraphics[width, height, ...]{}
\setkeys{Gin}{width=\maxwidth,height=\maxheight,keepaspectratio}
\IfFileExists{parskip.sty}{%
\usepackage{parskip}
}{% else
\setlength{\parindent}{0pt}
\setlength{\parskip}{6pt plus 2pt minus 1pt}
}
\setlength{\emergencystretch}{3em}  % prevent overfull lines
\providecommand{\tightlist}{%
  \setlength{\itemsep}{0pt}\setlength{\parskip}{0pt}}
\setcounter{secnumdepth}{5}
% Redefines (sub)paragraphs to behave more like sections
\ifx\paragraph\undefined\else
\let\oldparagraph\paragraph
\renewcommand{\paragraph}[1]{\oldparagraph{#1}\mbox{}}
\fi
\ifx\subparagraph\undefined\else
\let\oldsubparagraph\subparagraph
\renewcommand{\subparagraph}[1]{\oldsubparagraph{#1}\mbox{}}
\fi

%%% Use protect on footnotes to avoid problems with footnotes in titles
\let\rmarkdownfootnote\footnote%
\def\footnote{\protect\rmarkdownfootnote}

%%% Change title format to be more compact
\usepackage{titling}

% Create subtitle command for use in maketitle
\newcommand{\subtitle}[1]{
  \posttitle{
    \begin{center}\large#1\end{center}
    }
}

\setlength{\droptitle}{-2em}

  \title{A tao of programming}
    \pretitle{\vspace{\droptitle}\centering\huge}
  \posttitle{\par}
    \author{Dan MacLean}
    \preauthor{\centering\large\emph}
  \postauthor{\par}
      \predate{\centering\large\emph}
  \postdate{\par}
    \date{2018-09-17}

\usepackage{booktabs}

\begin{document}
\maketitle

{
\setcounter{tocdepth}{1}
\tableofcontents
}
\hypertarget{what-is-programming}{%
\chapter{What is programming}\label{what-is-programming}}

Programming is the task of creating instructions so that a computer can
perform a task for you.

At the highest level programming has two steps: design a solution to a
problem then encode the solution so a computer can do the work. The
first part is a challenge for creative logic, the second a challenge for
language. Programming a computer takes more of our imagination and
creativity than it does of our cold, hard logic or mathematics.

This guide is intended to help a beginner reach an understanding of how
to design a solution to a problem in such a way that it can later be
translated into a programming language. No coding in a programming
language is demonstrated in this guide, instead I focus on the building
blocks of solutions for programming: sequences, loops, events,
conditions, and I focus on the key computational thinking practices:
experimentation and iteration, testing, re-use and abstraction. In this
guide we use a graphical tool called Scratch to create and apply these
concepts in a computer.

The guide aims to teach an understanding of the way that programs are
built, before we understand how programs are encoded in a language.

\hypertarget{getting-started}{%
\chapter{Getting Started}\label{getting-started}}

\begin{quote}
This is not going to go the way you think.
\end{quote}

Be prepared, this is not a typical programming course. It will be a
largely self led exploration of concepts and ideas, not a slog through
syntax and rote copying. If you've seen a programming guide before,
forget it. Here your best tools will be your imagination, a notebook, a
friend and your powers of reflection and criticism. This guide will be
your research partner in the exploration helping you to reveal the
fundamental concepts of programming.

For this course you will need.

\begin{enumerate}
\def\labelenumi{\arabic{enumi}.}
\tightlist
\item
  A computer with an internet connection
\item
  A notebook
\item
  A friend (actually this is optional, but will be helpful).
\end{enumerate}

\hypertarget{reflection-and-critique-as-programming-tools}{%
\section{Reflection and critique as programming
tools}\label{reflection-and-critique-as-programming-tools}}

The extent to which programming as a professional tool is a
collaborative effort isn't widely appreciated. Stereotypes of `hackers'
in media abound - and this stereotype suggests that building a program
comes from heroic spurts of inspiration and esoteric knowledge. More
realistically building a program is a slow process that involves
thinking about a problem, weighing up different potential solutions and
expressing them. Reflecting on these things and criticising them (in the
strict neutral sense of evaluating and assessing dispassionately) are
really important tools for advancing the program.

Many parts of this guide will involve reflection and discussion, so
having someone you feel comfortable discussing stuff with will be
helpful. It's not absolutely necessary, though. You can do all of this
guide by yourself if you'd prefer.

\hypertarget{scratch---our-creativity-tool}{%
\section{Scratch - Our Creativity
Tool}\label{scratch---our-creativity-tool}}

Scratch is a free graphical computer program for creating media
projects. It is available at \url{http://scratch.mit.edu}

With it you can create a wide variety of interactive projects -
animations, games etc. Hundreds of thousands of people use Scratch
across the world, including primary school children and Harvard computer
science undergraduates learning to program. It's designed to be
accessible yet complete. It encompasses all the key concepts we'll need
to understand programming. Take a look at this
\href{https://vimeo.com/65583694}{introductory video}.

Let's investigate Scratch!

\hypertarget{for-you-to-do}{%
\section{For you to do}\label{for-you-to-do}}

These tasks may seem trivial, hopefully they'll seem playful. Have some
fun with them.

\begin{enumerate}
\def\labelenumi{\arabic{enumi}.}
\tightlist
\item
  Sign-up for Scratch \url{https://scratch.mit.edu/}
\item
  Browse some starter projects
  \url{https://scratch.mit.edu/starter_projects/}
\item
  In your notebook, sketch ideas for three different Scratch projects
  you would like to create.
\item
  Go to Scratch and make the Scratch cat do something surprising.
\end{enumerate}

\hypertarget{wait-what}{%
\subsection{Wait, what?}\label{wait-what}}

Yep, that number four does say to go and do something in Scratch.

\hypertarget{but-you-havent-shown-us-how-to-use-it-are-we-supposed-to-just-go-and-do-it-what-are-we-learning-here}{%
\subsection{But you haven't shown us how to use it? Are we supposed to
just go and do it? What are we learning
here?}\label{but-you-havent-shown-us-how-to-use-it-are-we-supposed-to-just-go-and-do-it-what-are-we-learning-here}}

Glad you asked! The object here would be for you to identify the stomach
churning desperation that comes from not knowing how to solve something,
yet being committed to doing so. And with that burning in your gut you
manage not to be paralysed by the darkness of ignorance and still manage
to claw your way into the light.

A lot of the time with programming you're not going to know exactly what
to do upfront. This reaches out to your inquisitiveness and curious
spirit. Just give it a go - you can't break anything or go wrong.

Look at this way, if you're sitting there thinking you don't know how to
do it, anything you do will be a surprise!

You can do this, I believe in you.

Here's a \href{worksheets/scratch_surprise.pdf}{starter sheet} if you
would like a \emph{little} hint.

\hypertarget{s}{%
\subsection{S}\label{s}}

\hypertarget{diving-in}{%
\chapter{Diving In}\label{diving-in}}

Now let's talk details.

\bibliography{book.bib}


\end{document}
